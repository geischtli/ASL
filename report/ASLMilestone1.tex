\documentclass[11pt]{article}
\usepackage[a4paper, portrait, margin=1in]{geometry}



\begin{document}

\title{Advanced Systems Lab (Fall'15) -- First
Milestone}

\author{Name: \emph{Your name}\\Legi number: \emph{Your legi
number}}

\date{
\vspace{4cm}
\textbf{Grading} \\
\begin{tabular}{|c|c|}
\hline  \textbf{Section} & \textbf{Points} \\
\hline  1.1 &  \\
\hline  1.2 &  \\
\hline  1.3 &  \\
\hline  2.1 &  \\
\hline  2.2 &  \\
\hline  2.3 &  \\
\hline  3.1 &  \\
\hline  3.2 &  \\
\hline  3.3 &  \\
\hline  3.4 &  \\
\hline  3.5 &  \\
\hline  3.6 &  \\
\hline \hline Total & \\
\hline
\end{tabular}
}

\maketitle

\newpage

\section*{Notes on writing the report}

The report for first milestone not need to be extensive but it must be concise,
complete, and correct.
Conciseness is important in terms of content and explanations, focusing on what
has been done and explanations of the results. A long report is not necessarily
a better report, especially if there are aspects of the design or the experiment
s that remain unexplained. Completeness implies that the report should give a co
mprehensive idea of what has been done by mentioning all key aspects of the desi
gn, experiments, and analysis. Aspects of the system, be it of its design or of
its behavior, that remain unexplained detract from the credibility of the report
. Correctness is expected in terms of the explanations being logical and correla
te with the numbers in the experiments and the design.

Remember that this is a report about the system you have designed and built, abo
ut the experiments you have performed, and about how you interpret the results o
f the experiments and map them to your design and implementation. There is no un
ique way to do the report but we provide you in this template with a structure t
hat covers all important aspects of the project. Please do not contact us seekin
g confirmation and assurances about, e.g., whether the report is sufficient, you
r interpretation of the data, validation of concrete aspects of your design, or
whether you have done enough experiments. Making those decisions is your job and
 part of what the course will evaluate.

The report will be graded together with the code and data submitted. You might b
e called for a meeting in person to clarify aspects of the report or the system
and to make a short presentation of the work done. By submitting the report, the
 code, and the data, you confirm that you have done the work on your own, the co
de has been developed by yourself, the data submitted comes from experiments you
r have done, you have written the report on your own, and you have not copied ne
ither code nor text nor data from other sources.

A passing grade for the milestone requires at the very minimum:
\begin{itemize}
\item A working system
\item Consistent experimental results
\item Measurements of the entire system
\item Measurements of each component (database and middleware)
\item In depth analysis of either database or middleware
\item Solid and credible explanations of the design, results, experiments and be
havior of the system
\end{itemize}

\section*{Formatting guidelines}
While you can use any text processor of your choice for writing the report, plea
se conform to the following formatting rules:
\begin{itemize}
\item  We expect you to submit \textbf{a single PDF that has the same section st
ructure as this template} (if you use this file, you should remove this page wit
h notes, and the short description provided by us at the beginning of sections).
\item  The main text should be in \textbf{single-column format with 11pt font on A4 paper}. In case you don't start with one of the files provided by us,
\textbf{for margins use 2.54 cm (1 inch) on all sides}.
\end{itemize}


\section{System Description}\label{sec:system-description}

\subsection{Database}\label{sec:database}

Length: 1-2 pages

Start by explaining the schema of the database and the indexes used to
speed up data access. Describe the interface to the database (queries
and stored procedures).

Make sure to explain the design in terms of what you wanted to achieve,
what decisions you took and what is the expected behavior.

Include baseline performance characteristics of the database (max
throughput, response time, and scalability).

\subsubsection{Schema and Indexes}\label{sec:schema-and-indexes}

\subsubsection{Stored Procedures}\label{sec:stored-procedures}

\subsubsection{Design decisions}\label{sec:design-decisions}

\subsubsection{Performance characteristics}\label{sec:performance-characteristic
s}

\subsection{Middleware}\label{sec:middleware}

Length: 1-2 pages

Explain the design from a high-level point of view, highlighting what
you wanted to achieve, design decisions, expected behavior.

Then go into more detail on how the middleware connects to the database
and clients, and how queuing is implemented.

Show what are the performance characteristics of the middleware
(i.e.~throughput, latency, scalability).

\subsubsection{Design overview}\label{sec:design-overview}

\subsubsection{Interfacing with clients}\label{sec:interfacing-with-clients}

\subsubsection{Queuing and Connection pool to database}\label{sec:queuing-and-co
nnection-pool-to-database}

\subsubsection{Performance characteristics}\label{sec:performance-characteristic
s-1}

\subsection{Clients}\label{sec:clients}

Length: 2-3 pages

Explain the interface of the clients to your messaging system and their
high level design, including the ways you have instrumented the code for
debugging and benchmarking purposes.

Provide a detailed description of the workloads used later in the report
(operation mix, starting and ending state of the database, assumptions
on workload behavior). Explain how the load was generated (include
baselines on load generation speed) and how the clients were deployed.

Which are the sanity checks in place for ensuring correct load
generation and validity of responses?

\subsubsection{Design and interface}\label{sec:design-and-interface}

\subsubsection{Instrumentation}\label{sec:instrumentation}

\subsubsection{Workloads and deployment}\label{sec:workloads-and-deployment}

\subsubsection{Sanity checks}\label{sec:sanity-checks}

\section{Experimental Setup}\label{sec:experimental-setup}

Length: 1-2 pages

Explain the overall design of the complete system and list the
configurations (number of middlewares, number of clients, types of
machines, communication patterns) corresponding to the main workloads.

Describe the mechanisms for deploying the system for experiments and the
way performance numbers are gathered and processed. Make the description
so that someone unfamiliar with your system can replicate the steps, and
reference the different script files you submit as code in the SVN
repository.

\subsection{System Configurations}\label{sec:system-configurations}

\subsection{Configuration and Deployment mechanisms}\label{sec:configuration-and
-deployment-mechanisms}

\subsection{Logging and Benchmarking mechanisms}\label{sec:logging-and-benchmark
ing-mechanisms}

\section{Evaluation}\label{sec:evaluation}

Length: up to 10 pages

In this section we expect to see the different experiments you ran to
exercise the system, and with each experiment we expect a clear
description of the system configuration used, the hypothesis on behavior
and the explanation of the behavior observed (in terms of the different
design decisions taken beforehand) -- \emph{missing either of these for
an experiment might make you lose all points for that given experiment!}
Keep in mind that for a good explanation of the results of an experiment
you might have to use one or more methods of data analysis presented in
the lecture and in the book.

See below for a short description on what each part should contain.

\subsection{System Stability}\label{sec:system-stability}

To prove that your system functions correctly and that it is stable
include the trace of a 30 minute run, plotting both response time and
throughput. Use at least 30 clients (sending and receiving data), 2
middlewares and a non-empty database.

\subsection{System Throughput}\label{sec:system-throughput}

Measure the maximum throughput of the system (describe the exact
configuration and workload, and the reasoning behind choosing these
particular ones) and show the average response time for this experiment.

\subsection{System Scalability}\label{sec:system-scalability}

Explain the different configurations used to explore the scalability of
your system, and the outcomes of these experiments in terms of
throughput and response times. The main goal of this subsection is to
define the ranges in which your system operates best.

\subsection{Response Time Variations}\label{sec:response-time-variations}

Report and analyze how the response times change in the system with
different message sizes, different number of clients and different
number of middleware nodes.

\subsection{$2^k$ Experiment}\label{sec:k-experiment}

Conduct a 2\^{}k analysis of your system (aim at exploring non-obvious
interactions of parameters). Use the methods learned in this lecture to
conduct the detailed analysis.

\subsection{Conclusion}\label{sec:conclusion}

To conclude the report summarize the behavior of the system in terms of
the design and the representative workloads. Finally, outline in a few
points what would you do differently if you could design the system
anew.

\end{document}